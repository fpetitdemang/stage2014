\section{Tree pattern matching}
Dans le cadre de notre sujet on explore différentes pistes pour la détection d'objet type définit à priori. Ses problèmes sont étudiés dans le domaine de recherche d'information (RI). 
Une approche de ce domaine est la comparaison d'arbre. Cette approche est apparenté aux problématiques de correspondance de motif dans les structures arborescente (tree pattern matching). De manière analogue aux problèmes de correspondance de motif dans une chaîne (string pattern matching) qui prend entrée un motif et un texte et qui produit en sortie la localisation d'une sous-chaîne de caractères du texte en entrée correspondant aux motifs recherchés. Dans le tree pattern matching, le motif et  le texte sont des structures arborescentes. Le problème de recherche de motif (pattern matching) consiste à trouver tous les sous-arbres du texte isomorphe avec le pattern en entrée. 
Dans la littérature ont trouve plusieurs mesures de distance entre deux arbres utiles à la comparaison d'arbre. 
