\documentclass[9pt]{beamer}
 
\usepackage[utf8]{inputenc} 
\usepackage[T1]{fontenc}
\usepackage{lmodern}
\usepackage{graphicx}
\usepackage[french]{babel}
\usepackage{tikz}
\usepackage{listings}
\usepackage{algpseudocode}
 %\useoutertheme[subsection=false]{smoothbars}
 
 
\usepackage{graphicx}
\usepackage{caption}
\usepackage{subcaption}

\usepackage{tikz}
\usetikzlibrary{trees}
\usetikzlibrary{positioning,shadows,arrows}

\usetheme{Warsaw}
%\usetheme{Madrid}
%\usetheme{Frankfurt}
\setbeamertemplate{navigation symbols}{\insertframenumber/\,\inserttotalframenumber}

\begin{document}
%\hspace*{1cm} \insertframenumber\,/\,\inserttotalframenumber
\title[Inférence de la structure d'une page web]{Inférence de la structure d'une page web en vue d'améliorer son accessibilité}
\subtitle[\ldots]{Encadré par : Y. Bonavero, M. Huchard et M. Meynard}
\author[Franck PETITDEMANGE]{Franck PETITDEMANGE}
\institute[LIRMM]{\includegraphics[scale=0.4]{img/logo-bergerlevrault.jpg}\hspace{0.5cm}\includegraphics[scale=0.2]{img/logo-lirmm.jpg}\hspace{0.5cm}\includegraphics[scale=0.3]{img/logo-cnrs.jpg}}
\date{26 juin 2014}
%\setbeamertemplate{footline}[page number]




\begin{frame}
\titlepage
\end{frame}

\begin{frame}
  \frametitle{Sommaire}
  \tableofcontents[hideothersubsections]
\end{frame}

\section{Introduction} 
\begin{frame}
  \frametitle{Sommaire}
  \tableofcontents[currentsection, hideothersubsections]
\end{frame}

\begin{frame}
	\frametitle{Accessibilité du web}
	\framesubtitle{Un enjeu sociétal important}
	\begin{block}{Définition}
	\textbf{Accessibilité :} capacité d'accéder aux informations contenues dans une page et d'interagir avec.
	\end{block}
	\begin{block}{Problèmes d'accessibilité (spécifique aux basses visions)}
		\begin{itemize}
			\item Surcharge visuelle
			\item Police de caractères
			\item Contraste de couleur
		\end{itemize}
	\end{block}
\end{frame}

\begin{frame}
\frametitle{Accessibilité du web}
\framesubtitle{Besoin de comprendre la structuration d'une page}
	%\begin{columns}
	%\begin{column}{5cm}
	\begin{block}{Problèmes des outils d'accessibilité (spécifique aux basses visions)}
	\begin{itemize}
		\item Pas de traitement des couleurs locales
		\item Pas de prise en compte des profils utilisateurs
	\end{itemize}
	\end{block}
	\begin{block}{Besoin}
	\begin{itemize}
		\item Comprendre les informations structurant une page web pour réaliser une adaptation basée sur cette structure
	\end{itemize}
	\end{block}
	%\end{column}
	%\begin{column}{5cm}
	%\fbox{\includegraphics[scale=0.2]{img/page_berger.png}}
	%\end{column}
	%\end{columns}
\end{frame}



\begin{frame}
	\frametitle{Page web}
	\begin{columns}
	\begin{column}{5cm}
		\begin{block}{Page web}
		\begin{itemize}
			\item Technologies : HTML/CSS/Javascript
			\item Contenu hétérogène décrit par différentes structures logiques
		\end{itemize}
		\end{block}
		%\begin{block}{}
		%\end{block}
	\end{column}
	\begin{column}{5cm}
	\fbox{\includegraphics[scale=0.2]{img/segmentation_page_berger.png}}
	\end{column}
	\end{columns}
\end{frame}

\begin{frame}
	\frametitle{}
	\begin{block}{}
	Comment inférer les différentes structures logiques dans une page web?
	\end{block}
	\begin{block}{Difficultés}
		\begin{itemize}
			\item Manque d'expressivité de HTML 4
			\item Pas de construction standard des structures logiques
			%\item Syntaxe de HTML 4 autorise des constructions non conformes au W3C
			\item Écart entre la structure DOM et l'affichage dans un navigateur
		\end{itemize}
	\end{block}
	\begin{block}{Approche}
		\begin{itemize}
			\item Étude des langages de publication de page web
			\item Étude des techniques d'extraction de structure d'une page
		\end{itemize}
	\end{block}
\end{frame}


\section{État de l'art}

\begin{frame}
  \frametitle{Sommaire}
  \tableofcontents[currentsection, hideothersubsections]
\end{frame}

\subsection{Étude des Langages de publication}
\begin{frame}[fragile]
\frametitle{Évolution de la sémantique (1/2)}
\begin{columns}
	\begin{column}{5.5cm}
\begin{lstlisting}[frame=single, language=html]
<ul class='menu'>
 <li><a href=".">l1</li>
 <li><a href=".">l2</li>
</ul>
<p class='menu'>
 <a href=".">l1</a>
 <a href=".">l2</a>
</p>
\end{lstlisting}
	\end{column}
	\begin{column}{5cm}
	\includegraphics[scale=0.4]{img/architecture_HTML4.jpg}	
	\end{column}
\end{columns}
	\begin{block}{HTML 4}
	\begin{itemize}
		\item Peu de sémantique
		\item Diversité de représentation
		\item Structure logique implicite
	\end{itemize}
	\end{block}
\end{frame}

\begin{frame}
\frametitle{Évolution de la sémantique (2/2)}
\begin{columns}
	\begin{column}{5cm}
	\begin{block}{HTML 5}
		\begin{itemize}
			\item Structure logique explicitée
			\item Sémantique pour décrire l'interface de la page limitée
		\end{itemize}
	\end{block}
	\begin{block}{ARIA}
		\begin{itemize}
			\item Ontologie d'une interface graphique
			\item Trop élaborée pour nos besoins mais est plus expressif
		\end{itemize}
	\end{block}
	\end{column}
	\begin{column}{5cm}
	\temporal<2->{\includegraphics[scale=0.5]{img/architecture_HTML5.jpg}}{\includegraphics[scale=0.5]{img/architecture_ARIA.jpg}}
	
	\end{column}
\end{columns}
\end{frame}


\begin{frame}
\frametitle{CSS}
\framesubtitle{Un langage de mise en forme}
\begin{columns}
	\begin{column}{5cm}
	\begin{block}{Propriétés de mise en forme :}		
		\begin{itemize}
			\item avant-plan/arrière-plan
			\item police
			\item ...
		\end{itemize}
	\end{block}
	\begin{block}{Mécanisme de positionnement}
		\begin{itemize}
			\item relatif
			\item absolu
			\item flottant
		\end{itemize}
	\end{block}
	\end{column}
	\begin{column}{5cm}
	\end{column}
\end{columns}\end{frame}

\begin{frame}
\frametitle{Synthèse}
\begin{block}{}
HTML 4 langage actuellement le plus exploité. Les inconvénients sont :
\begin{itemize}
	\item la diversité de représentation d'une même structure logique
	\item la faible expressivité au regard des concepts décrits dans les pages web
	\item l'absence de structuration explicite
\end{itemize}
\end{block}
\begin{block}{Notre approche}
\begin{itemize}
	\item Proposer un Méta-modèle concrétisant mieux les concepts des pages et permettant de s'abstraire de la diversité de représentation des structures 
\end{itemize}
\end{block}
\end{frame}

\subsection{Étude de méthodes d'extraction de structure}
\begin{frame}
\frametitle{Mapping}
\framesubtitle{\textit{(Vieira et al., A fast and robust method for web page template detection and removal)}}
\begin{columns}
	\begin{column}{5cm}
	\begin{block}{Mapping descendant restrictif}
		Permet de faire correspondre les plus grandes sous-structures communes entre deux arbres.\\
		%Conserve \textbf{l'ordre des frères}, \textbf{la relation d'héritage}, \textbf{mappe pas les descendants de couple n'ayant pas les mêmes étiquettes}
	\end{block}
	\begin{block}{Idée}
		Identifier les structures logiques des pages web par correspondance
	\end{block}
	\end{column}
	\begin{column}{5cm}
		\includegraphics[scale=0.3]{img/mapping.jpg}
	\end{column}
\end{columns}
\end{frame}

\begin{frame}
\frametitle{Segmentation}
\framesubtitle{par pattern de présentation \textit{(Milos Kovacevic et al., Recognition of Common Areas in a Web Page Using Visual Information: a possible application in a page classification)}}
\begin{columns}
	\begin{column}{5cm}
	\begin{block}{Observation}
		Les concepteurs de page web suivent approximativement les mêmes schémas de présentation
	\end{block}
	\begin{block}{Idée}
		Regrouper les n\oe{}uds du DOM de la page suivant leurs coordonnées après la mise en page par le navigateur
	\end{block}
	\end{column}
	\begin{column}{5cm}
		\includegraphics[scale=0.3]{img/segmentation-pattern.jpg}
	\end{column}
\end{columns}
\end{frame}

\begin{frame}
\frametitle{Segmentation}
\framesubtitle{par densitométrie textuelle \textit{(Kohlsch{\"u}tter et al, A densitometric approach to web page segmentation)}}

\begin{block}{Étape 1 : identification de segments de petites tailles}	
La page est vue comme une séquence de caractères entrelacés identifiés par des balises HTML. Les segments sont calculés d'après les variations dans le rythme des séquences pour lesquels on calcule une densité textuelle.\\
Exemple : \\
Ici deux segments seront calculés \\
\textit{<a>lienA</a>lienB<a>lienC</a><p>un paragraphe</p>}
\end{block}

\begin{block}{Étape 2 : grossissement successif des segments par fusion}
Les segments contigus dont la densité textuelle est proche sont fusionnées successivement.
\end{block}

\end{frame}

\begin{frame}
\frametitle{Segmentation}
\framesubtitle{par indice visuel \textit{(Cai et al., Extracting content structure for web pages based on visual representation)}}
\includegraphics[scale=0.3]{img/segmentation-vips.png}
\begin{block}{Idée}
Regroupement par indice visuel
\end{block}
\end{frame}

\begin{frame}
\frametitle{Synthèse}
\begin{block}{Les inconvénients}
	\begin{itemize}
		\item Le \emph{Mapping} ne permet pas d'extraire la structure globale de la page
		\item La segmentation par pattern est trop dépendante de la présentation de la page
		\item La segmentation par densitométrie ne prend pas en compte les écarts possibles entre le DOM et le rendu final
		\item Le calcul des séparateurs dans l'approche par indice visuel est une opération coûteuse $O(n^2)$
	\end{itemize}
\end{block}
\begin{block}{Notre approche : segmentation visuelle}
Propose un \textbf{découpage global} de la page, \textbf{indépendant des patterns de présentation} et permet un \textbf{découpage fin} dans la structure d'une page.
\end{block}
\end{frame}

\section{Réalisation}
\begin{frame}
  \frametitle{Sommaire}
  \tableofcontents[currentsection, hideothersubsections]
\end{frame}

\subsection{Méta-modèle}

\begin{frame}
	\frametitle{Approche générale}
	\framesubtitle{Une approche d'Ingénierie Dirigée par les Modèles}
	\begin{columns}
		\begin{column}{4cm}
			\begin{block}{Avantages}
				\begin{itemize}
					\item Meilleure expression des préférences
					\item Écriture d'adaptation indépendante des langages
				\end{itemize}
			\end{block}
		\end{column}
		\begin{column}{6cm}
		\includegraphics[scale=0.3]{img/workflow-adaptation.jpg}
		\end{column}		
	\end{columns}
\end{frame}






\begin{frame}
\frametitle{Méta-modèle intermédiaire}
\framesubtitle{Éléments structurels}
\begin{figure}
\centering
\includegraphics[scale=0.335]{img/metamodele_structure.png}
\end{figure}
\end{frame}

\begin{frame}
\frametitle{Méta-modèle intermédiaire}
\framesubtitle{Éléments d'interaction}
\begin{figure}
\centering
\includegraphics[scale=0.4]{img/metamodele_widget.png}
\end{figure}
\end{frame}

\begin{frame}
\frametitle{Méta-modèle intermédiaire}
\framesubtitle{Éléments de mise en forme}
\begin{figure}
\centering
\includegraphics[scale=0.4]{img/metamodele_CSS.png}
\end{figure}
\end{frame}

\subsection{Extraction structure}
\begin{frame}
\frametitle{Démarche générale}
\begin{algorithmic}
%\State $arbreIntermediaire$

\Function{ConstStructLog}{$noeudDom$, $noeudStructInter$}
	\State $poolNoeuds\gets 0$\\ 
	\Call {DivisionDom}{$noeudDom$, $poolNoeuds$}
	\For {i in $poolNoeuds$}\\
		\Call{AppendChild}{$arbreIntermediaire$, $poolNoeud[i]$}\\
		\Call{ConstStructLog}{$poolNoeud[i]$,$arbreIntermediaire$}
	\EndFor
\EndFunction
\end{algorithmic}
\vspace{0.5cm}
\begin{algorithmic}
\Function{DivisionDom}{$noeudDom$, $poolNoeuds$}
	\If {\Call{Divisible}{$noeudDom$} == TRUE}
		\ForAll{Enfant de $noeudDom$}\\
			\Call{DivisionDom}{$noeudDom$, $poolNoeuds$}		
		\EndFor
	\Else
	\State $poolNoeuds\gets poolNoeuds+noeudDom$
\EndIf
\EndFunction
\end{algorithmic}
\end{frame}

\begin{frame}
\frametitle{Concepts et heuristiques}
\begin{block}{Concepts}
	\begin{description}
		\item[N\oe{}uds conteneurs :]propriété CSS display égale à \emph{bloc} 
		\item[N\oe{}uds de mise en forme :]propriété CSS display à \emph{inLine}
		\item[Taille d'un n\oe{}ud :] taille relatif à sa position dans le DOM et à la somme du poids des n\oe{}uds qu'il contient
		\item[Distance visuelle :] signale les changements de propriété de mise en forme des n\oe{}uds
	\end{description}
\end{block}
\begin{block}{Règles}
\begin{description}
	\item[Règle 2 :] si le n\oe{}ud possède un seul enfant et que cet enfant n'est pas un n\oe{}ud de données alors on parcourt ce n\oe{}ud
	\item[Règle 4 :] si l'un des enfants du n\oe{}ud est un n\oe{}ud conteneur et que sa taille est supérieure à un certain seuil alors on parcours ce n\oe{}ud
	\item[Règle 8 :] si la fonction de distance visuelle est vérifiée pour l'un des n\oe{}uds enfants alors on parcourt ce n\oe{}ud et on extrait les enfants qui vérifient la fonction
\end{description}
\end{block}
\end{frame}

\begin{frame}
	\frametitle{Processus d'extraction}
	\framesubtitle{Segmentation globale page Berger-Levrault}
	\begin{columns}
		\begin{column}{5cm}
			\fbox{\includegraphics[scale=0.2]{img/page_berger.png}}
		\end{column}
		\begin{column}{5cm}
			\fbox{\includegraphics[scale=0.2]{img/segmentation_page_berger.png}}
		\end{column}
	\end{columns}
\end{frame}

\begin{frame}
	\frametitle{Processus d'extraction}
	\framesubtitle{Segmentation locale page Berger-Levrault}

		\fbox{\includegraphics[scale=0.3]{img/banner_berger.png}}
		\vspace{1cm}
		\fbox{\includegraphics[scale=0.3]{img/segmentation_banner_berger.png}}	

\end{frame}

\subsection{Annotation structure}
\begin{frame}
\frametitle{Approche par fonctionnalité}
\framesubtitle{Construction de fonction d'annotation \textit{(Chen et al., Function-based object model towards website adaptation)}}
\begin{block}{}
Classe de n\oe{}uds :
\begin{itemize}
	\item Informatif
	\item Navigation
	\item Interaction
	\item Décoration
\end{itemize}
Chaque classe possède les propriétés : 
\begin{itemize}
\item Hyperlien
\item Sémantème
\item Décoration
%\item Alignement
\item \textbf{Position}
\end{itemize}
\end{block}

\end{frame}

%\begin{frame}
%\frametitle{Approche par fonctionnalité}
%\framesubtitle{fonctions de détection proposées}
%\begin{block}{}
%	\begin{description}
%		\item[Banner :]
%		\item[ContentInfo :]
%	\end{description}
%\end{block}
%\end{frame}

\begin{frame}

\begin{tikzpicture}[
    fact/.style={rectangle, draw=none, rounded corners=1mm, fill=blue, drop shadow,
        text centered, anchor=north, text=white},
    state/.style={rectangle, draw=none, fill=orange, drop shadow,
        text centered, anchor=north, text=white},
    fbo/.style={rectangle, draw=none, fill=red, drop shadow,
        text centered, anchor=north, text=white},
    leaf/.style={circle, draw=none, fill=red, circular drop shadow,
        text centered, anchor=north, text=white},
    level distance=0.5cm, growth parent anchor=south
]

\node [state] {...} [->, sibling distance=3cm]
		child{ node [state] {CO}
        child{
            node [fbo] {BO}}
        child{
        		node [state] {CO}
        		child { node [fbo] {BO}}
        		child { node [state] {CO}
        				child { node [fbo] {BO}}
        				child { node [fbo] {BO}}
        				child { node [fbo] {BO}}}}
		child{
            node [fbo] {BO}}}		            
;

\end{tikzpicture}
\end{frame}


\begin{frame}

\begin{tikzpicture}[
    fact/.style={rectangle, draw=none, rounded corners=1mm, fill=blue, drop shadow,
        text centered, anchor=north, text=white},
    state/.style={rectangle, draw=none, fill=orange, drop shadow,
        text centered, anchor=north, text=white},
    fbo/.style={rectangle, draw=none, fill=red, drop shadow,
        text centered, anchor=north, text=white},
    leaf/.style={circle, draw=none, fill=red, circular drop shadow,
        text centered, anchor=north, text=white},
    level distance=0.5cm, growth parent anchor=south
]

\node [state] {...} [->, sibling distance=3cm]
		child{ node [state] {BAN}
        child{
            node [fbo] {INFO-IMG}}
        child{
        		node [state] {CO}
        		child { node [fbo] {INTER}}
        		child { node [state] {NAV}
        				child { node [fbo] {NO}}
        				child { node [fbo] {NO}}
        				child { node [fbo] {NO}}}}
		child{
            node [fbo] {INFO-TITRE}}}		            
;

\end{tikzpicture}
\end{frame}



\section{Conclusion}
\begin{frame}
  \frametitle{Sommaire}
  \tableofcontents[currentsection, hideothersubsections]
\end{frame}
%\subsection{Résultats}
\begin{frame}
	\frametitle{Difficultés}
	\begin{block}{Difficultés}
		\begin{itemize}
			\item Domaine de recherche éloigné des connaissances de l'équipe d'accueil
			\item Définir la problématique par rapport à la question de recherche (recherche de motifs d'intérêt dans un arbre DOM)
			%\item Recherche d'articles traitant de la problématique
		\end{itemize}
	\end{block}
\end{frame}

%\subsection{Difficultés et perspectives}
\begin{frame}
	\frametitle{Résultats}
	\begin{block}{Résultats}
		\begin{itemize}
			\item Proposition d'une approche IDM et élaboration d'un méta-modèle
			\item État de l'art sur les langages de publication et des techniques d'extraction de structure de page web
			\item Adaptation et implémentation d'une méthode pour extraire les structures d'une page
			\item Proposition de pistes pour annoter les structures extraites et début d'implémentation  
		\end{itemize}
	\end{block}
\end{frame}

\begin{frame}
	\frametitle{Perspectives}
		\begin{block}{Perspectives à court terme}
		\begin{itemize}
			\item Évaluation de la méthode d'extraction de structure
			\item Implémentation, évaluation et élargissement du processus d'annotation
		\end{itemize}
	\end{block}
	\begin{block}{Perspectives à long terme}
		\begin{itemize}
			\item Évaluation des modèles intermédiaires générés par le processus d'extraction et d'annotation sur un grand nombre de pages 
			\item Acquisition des préférences utilisateurs
			\item Intégrer les préférences utilisateurs dans le processus de transformation
		\end{itemize}
	\end{block}
\end{frame}


 
\end{document}